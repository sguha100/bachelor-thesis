\chapter{Automata without timing and relations on  these.}

\section{Labelled transition systems}

\begin{SCfigure}
  \centering
  \def\svgwidth{0.3\columnwidth}
  \input{lts01.pdf_tex}
  \caption{An example of a labelled transition system. Here, the
    states are $\{0, 1, 2, \ldots 7\}$ and the actions are $\{0,
    1\}$.}
  \label{lts01}
\end{SCfigure}

\begin{definition}
  \emph{Labelled Transition System}: A labelled transition system (LTS)
  \cite{Keller:1976:FVP:360248.360251} is an automaton which is
  described by
  \begin{itemize}
  \item $S$, a set of \emph{states} 
  \item $Act$, a set of \emph{actions}
  \item $\rightarrow \subseteq S \times Act \times S$, a \emph{transition
    relation}.
  \item optionally, $I \subseteq S$ ,a set of initial states. If there
    is exactly one initial state, then the LTS is said to be \emph{rooted}.
  \end{itemize}
\end{definition}

LTS are useful for describing the behaviour of untimed systems, and
serve as the foundation for the development of more complex models
such as CCS and timed automata. Thus, equivalences on LTS serve
as the theoretical foundation for many timed and time abstracted
equivalences on timed automata, and also have direct applications in
determining some of these equivalences in cases where timed automata
can be reduced to equivalent LTS.

\section{Equivalences on labelled transition systems}

\subsection{Strong bisimilarity}

\begin{SCfigure}
  \centering
  \def\svgwidth{0.5\columnwidth}
  \input{lts01quotient.pdf_tex}
  \caption{Strong bisimilarity quotient of the LTS in Figure~\ref{lts01}.}
\end{SCfigure}

A binary relation $R$ is a \textit{strong
  bisimulation} if and only if, for all $(s_1, s_2)$ $\epsilon$ $R$ and $a$ $\epsilon$ $Act .$\\
$\forall s_1' (s_1 \xrightarrow{a} s_1' \Rightarrow \exists s_2'
. (s_2 \xrightarrow{a} s_2' \wedge (s_1', s_2')$ $\epsilon$ $R ) )
\wedge $ \\
$\forall s_2' (s_2 \xrightarrow{a} s_2' \Rightarrow \exists s_1'
. (s_1 \xrightarrow{a} s_1' \wedge (s_1', s_2')$ $\epsilon$ $R ) )$

It can be shown that the union of
all strong bisimulations over the set of states is a strong
bisimulation. This binary relation is called \textit{strong
  bisimilarity}, denoted by $\sim$.

Strong bisimilarity between two states in an LTS implies,
intuitively, that any action performed by the one can be performed by
the other and vice versa.

An algorithm to evaluate strong bisimilarity was first given by
Kanellakis and Smolka \cite{kanellakis1990ccs}. A faster algorithm for
the special case having just one kind of action was given by Paige and
Tarjan \cite{paige1987three} and made general by Fernandez
\cite{fernandez1990implementation}.
