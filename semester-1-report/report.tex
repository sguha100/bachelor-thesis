\documentclass{article}

\begin{document}

\title{Tools and Algorithms for Deciding Timed Relations}

\author{Mihir Mehta\\
Department of Computer Science and Engineering,\\
Indian Institute of Technology, Delhi,\\
New Delhi, India.\\
\texttt{cs1090197@cse.iitd.ac.in}
}

\date{December 2012}

\maketitle

\begin{abstract}
This is a report summarising the author's project on their B Tech
Project for the academic year 2012-2013.
\end{abstract}

\section{Objectives}

\begin{itemize}

\item To develop a software toolkit that would enable users to verify
  various timed relations specifications and implementations expressed
  as timed automata.

  \begin{itemize}

  \item To gain an understanding of the theory related to labeled
    transition systems, CCS processes and timed automata by surveying
    relevant literature.

  \item To study tools already built by researchers for similar purposes.

  \item To develop the software in a modular way with modules for
    language specification and modules for implementations of utility
    algorithms.

  \item To implement algorithms for determining timed relations.

  \end{itemize}

\end{itemize}

\section{CCS processes}

CCS expression: Defined by the following grammar:
\begin{itemize}
\item $P::=K$
\item $P::=\alpha . P$
\item $P::=_{i \epsilon I} P_i$
\item $P::=P|Q$
\item $P::=P[f]$
\item $P::=P\L$
\end{itemize}

\section{Equivalences on LTS}

\subsection{Strong bismilarity}

\subsection{Weak bismilarity}

\subsection{Kanellakis and Smolka's algorithm}

\subsection{Fernandez' algorithm}

\section{Timed automata}

\section{Equivalences on Timed Automata}

\subsection{Timed bismilarity}

\subsection{Time abstracted bisimilarity}

\subsection{Weak timed bisimilarity}

\subsection{Regions and region graphs}

\subsection{Zones and zone graphs}

\end{document}
