\documentclass{beamer}
\usetheme{default}
\usepackage[algo2e]{algorithm2e}

\title{Tools and Algorithms for Deciding Relations on Timed Automata}
\subtitle{B Tech project, supervised by S Arun-Kumar, verification group}
\author{Mihir Mehta}
\institute[IITD]{
  Department of Computer Science and Engineering\\
  Indian Institute of Technology, Delhi\\[1ex]
  \texttt{cs1090197}
}
\date[May 2013]{May 12, 2013}

\begin{document}

%--- the titlepage frame -------------------------%
\begin{frame}[plain]
  \titlepage
\end{frame}

\AtBeginSection[]{

  \frame<beamer>{ 

    \frametitle{Outline}   

    \tableofcontents[currentsection,currentsubsection] 

  }

}

\section{Automata without timing and relations on them}

\begin{frame}{Labeled transition systems}
  \begin{definition}
    \emph{Labelled Transition System}: A labelled transition system (LTS)
    \cite{Keller:1976:FVP:360248.360251} is an automaton which is
    described by
    \begin{itemize}
    \item $S$, a set of \emph{states} 
    \item $Act$, a set of \emph{actions}
    \item $\rightarrow \subseteq S \times Act \times S$, a \emph{transition
      relation}.
    \item optionally, $I \subseteq S$ ,a set of initial states. If there
      is exactly one initial state, then the LTS is said to be \emph{rooted}.
    \end{itemize}
  \end{definition}
\end{frame}

\begin{frame}[allowframebreaks]{Relations on LTS}

\begin{definition}
\emph{Strong bisimulation}:A binary relation $R$ on the states of an
LTS is a strong bisimulation if and only if, for all
 $(s_1, s_2)$ $\epsilon$ $R$ and $a$ $\epsilon$ $Act .$\\
$\forall s_1' (s_1 \xrightarrow{a} s_1' \Rightarrow \exists s_2'
. (s_2 \xrightarrow{a} s_2' \wedge (s_1', s_2')$ $\epsilon$ $R ) )
\wedge $ \\
$\forall s_2' (s_2 \xrightarrow{a} s_2' \Rightarrow \exists s_1'
. (s_1 \xrightarrow{a} s_1' \wedge (s_1', s_2')$ $\epsilon$ $R ) )$
\end{definition}

\begin{definition}
It can be shown that the union of
all strong bisimulations over the set of states is a strong
bisimulation. This binary relation is called \emph{strong
  bisimilarity}, denoted by $\sim$.
\end{definition}

\end{frame}

\section{Timed automata and relations on them}

\begin{frame}{Timed Automata}
  \begin{definition}
    \emph{Timed Automaton}: A timed automaton
    \cite{Alur94atheory} over a finite set of clocks $C$
    and a finite set of actions $Act$ is a 4-tuple $(L, l_{0}, E, I)$.
    \begin{itemize}
    \item $L$ is a finite set of locations.
    \item $l_{0}$ is the initial location.
    \item $E \subseteq L \times B(C) \times Act \times 2^{C} \times L$
      is a finite set of edges.
    \item $I: L \rightarrow B(C)$ assigns invariants to each edge
      location.
    \item $B(C)$ is the set of clock constraints over C.
    \end{itemize}
  \end{definition}
\end{frame}

\section{Algorithms}

\begin{frame}[fragile]
  \frametitle{Creating the zone valuation graph}

  \begin{algorithm2e}[H]
    Initialise the queue $Q$ with a single element $(null, null, l_0)$\;
    Initialise the graph $zone\_graph$ with a single node $(l_0, v_0 \uparrow)$
    with an $\epsilon$ self-loop\;
    \While{$Q$ is not empty}{
      Dequeue $(l_{parent}, t, l_{child})$ from $Q$\;
      \If{$l_{parent} \neq null$}{
        \ForEach{zone $Z_{parent}$ of $l_{parent}$}{
          Add new zones to the zones of $l_{child}$ so that all zones
          reachable from $Z_{parent}$ are represented\;
          Abstract if necessary\;
          Update edges from $Z_{parent}$ to the new zones of $l_{child}$
          \If{new zones are created in $l_{child}$ or $l_{parent}$ is null}{
            \ForEach{outgoing transition $t'$ of $l_{child}$}{
              Enqueue $(l_{child}, t', \texttt{t'.target})$ in $Q$\;
            }
          }
        }
      }
      Set $new\_zone$\;
      \While{$new\_zone$}{
        Reset $new\_zone$\;
        \ForEach{transition $t$ in the timed automaton}{
          Split the zones of \texttt{t.source} to be stable with respect to the
          zones of \texttt{t.target}\;
          Update edges accordingly\;
          \If{new zones are created in \texttt{t.source}}{
            Set $new\_zone$\;
          }
        }
      }
    }
    Generate $zone\_valuation\_graph$ by applying Fernandez' algorithm to $zone\_graph$\;
    Return $zone\_graph$\;
  \end{algorithm2e}

\end{frame}

\begin{frame}{Zone valuation graph example}
  \begin{figure}
    \centering
    \def\svgwidth{0.6\columnwidth}
    \input{breaking2.pdf_tex}
    \caption{Timed automaton. Here, the states are \{0\}, the actions
      are \{a\}, and the clocks are \{X, Y\}.}
  \end{figure}
\end{frame}

\begin{frame}{Zone valuation graph example}
  \begin{figure}
    \centering
    \def\svgwidth{0.4\columnwidth}
    \input{breaking2-zones01.pdf_tex}
    \caption{Zones after one iteration.}
  \end{figure}
\end{frame}

\begin{frame}[shrink=20]{Zone valuation graph example}
  \begin{figure}
    \centering
    \def\svgwidth{0.7\columnwidth}
    \input{breaking2-zones02.pdf_tex}
    \caption{Zones after two iterations.}
  \end{figure}
\end{frame}

\begin{frame}[shrink=20]{Zone valuation graph example}
  \begin{figure}
    \centering
    \def\svgwidth{0.9\columnwidth}
    \input{breaking2-zones03.pdf_tex}
    \caption{Zones after three iterations without abstraction.}
  \end{figure}
\end{frame}

\begin{frame}[shrink=20]{Zone valuation graph example}
  \begin{figure}
    \centering
    \def\svgwidth{1.2\columnwidth}
    \input{breaking2-zones-abstracted.pdf_tex}
    \caption{Zones after three iterations with abstraction.}
  \end{figure}
\end{frame}

\begin{frame}[allowframebreaks,shrink=20]
  \frametitle{References}
  \bibliographystyle{splncs}
  \bibliography{presentation}
\end{frame}

\end{document}
